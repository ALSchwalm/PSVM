

\documentclass[12pt]{scrartcl}
\usepackage{graphicx}
\usepackage{pdfpages}

\begin{document}

\begin{titlepage}
\begin{center}

\vspace*{3.0 cm}
{\Huge Public Static Void Main}\\[0.7cm]
{\Large Requirements Document, Version 1}\\[3.8cm]
{\Large \today}\\[0.5cm]
Adam Schwalm\\
Eli Hunnicutt\\
Andrew LaFrance\\
Tyler Bertrand\\
Martin Kinsey\\[4.0cm]

Group Number: 5\\
Lab Instructor: Ajay Bandi
  

  
\end{center}
\end{titlepage}


\section{Team Responsibilities}

\begin{description}
\item [Requirements Presentation] - Martin Kinsey
\item [Design Presentation] - Adam Schwalm, Andrew LaFrance
\item [Final Presentation] - Tyler Bertrand, Eli Hunnicutt
\vspace{4mm}
\item [Adam Schwalm] - Backend Design, Server work
\item [Eli Hunnicutt] - Javascript Management, Design
\item [Andrew LaFrance, Martin Kinsey] - Graphic Design, Frontend Design work
\item [Tyler Bertrand] - Database Design, Management
\end{description}

\section{Account Information}

\begin{description}
\item [Group account username] - dcsp05
\item [Group account password] - keyboardkowboys
\end{description}

As the group project utilized python's built-in sqlite3 module, we did not connect
to the mySQL database. sqlite3 does not support passwords, and so there is no
password for the database.

\section{File Structure Layout}

The {\bfseries PSVM} folder contains all of the files used to run the server
as well as create the database. Running {\bfseries server.py} will start the
server while executing {\bfseries create\_database.py} will setup the database
in a default state.

The folder structure within {\bfseries PSVM} is as follows:

\begin{description}
\item [backend] - Holds the python files used by the server
\item [css] - Holds the css used to style the pages
\item [js] - Holds the javascript used to achieve some animations / redirects on the website
\item [templates] - Holds the html templates used when generating the dynamic pages
\item [docs] - Holds the LaTeX used to generate the various documents
\end{description}

\section{Database Design}

\textbf{users} \hfill \textbf{comments}\\*
*user\_id \hfill *comment\_id\\*
username \hfill \#thread\_id\\*
pass\_hash \hfill body\\*
email \hfill raw\\*
admin \hfill timestamp\\*
verified \hfill \#user\_id\\*
timestamp \\*
\\*
\textbf{categories} \hfill \textbf{code\_samples}\\*
*category\_id \hfill *sample\_id\\*
name \hfill language\\*
\hfill raw\\*
\hfill body\\*

\textbf{threads} \hfill \textbf{messages}\\*
*thread\_id \hfill *message\_id\\*
\#category\_id \hfill body\\*
title \hfill \#from\_id\\*
\#op\_id \hfill \#to\_id\\*
timestamp \hfill timestamp\\*

\section{Completed Requirements}

\begin{enumerate}
\item Create Account (Required)
\item Login to  Account (Required)
\item Browse Posts (Required)
\item Create Thread (Required)
\item Edit Posts (Required)
\item Create Subforum (Required)
\item Change Password (Required)
\item Edit Profile (Required)
\item View User Profile (Required)
\item Search Subforum (Required)
\item Upload User Image (Medium)
\item Embeded HTML (Medium)
\end{enumerate}

\section{Uncompleted Requirements}

Due to time constraints it was necessary to drop certain administrator
privileges, though these effects (such as banning uses) can still be
achieved by modifying the database directly.

Additionally, the 'signature' as dropped from the uses profile.

\section{Implementation}

The core of the implementation of the project was a mirror of the popular
django web framework for python. The requests are first sent to the core
dispatching function handled by python's WSGI module. Request objects
(a custom class) are then created from the WSGI request. This is mostly for
our convenience. These request objects are dispatched using regular expressions
to the appropriate module function. These functions construct the response
and reply to the request.

\end{document}
